\documentclass[a4paper,10pt]{article}

\usepackage[utf8]{inputenc}
\usepackage{t1enc}
\usepackage[spanish]{babel}
\usepackage[pdftex,usenames,dvipsnames]{color}
\usepackage[pdftex]{graphicx}
\usepackage{amsmath}
\usepackage{amsfonts}
\usepackage{amssymb}
\usepackage[table]{xcolor}
\usepackage[small,bf]{caption}
\usepackage{float}
\usepackage{subfig}
\usepackage{listings}
\usepackage{bm}
\usepackage{times}

\setcounter{secnumdepth}{5}

\begin{document}

\begin{titlepage}
	\thispagestyle{empty}
	\begin{center}
		\includegraphics[scale=0.7]{./images/itba.jpg}
		\vfill
		\Huge{Autómatas, Teoria de Lenguajes y Compiladores}\\
		\vspace{1cm}
		\huge{Trabajo Práctico Especial 2} \\
		\vspace{0.3cm}
		\huge{Title}
	\end{center}
	\vspace{2cm}
	\large{
		\begin{tabular}{lcr}
			Castiglione, Gonzalo & & 49138 \\
			Susnisky, Darío & & 50592 \\
			Ordano, Esteban & & 50753 \\
			Sturla, Martín & & 50684 \\
			\\ 
		\end{tabular}
	}
	\vfill
	\flushright{\today}
\end{titlepage}

\newpage

%%%%%%%%%%%%%%%%%%%%%%%%%%%%%%%%%%
%%%%%%%%% begin CONTENT %%%%%%%%%%
%%%%%%%%%%%%%%%%%%%%%%%%%%%%%%%%%%

	\thispagestyle{empty}
\tableofcontents

\newpage

\setcounter{page}{1}

\newpage

\section{Resumen}
El trabajo práctico consistió en generar un analizador sintáctico de partidas de ajedrez en formato \textit{PGN} 
para luego mostrar un tablero y poder hacer un seguimiento del partido.
\newpage

\section{Consideraciones realizadas}

\subsection{Formato PGN}

Como indica el enunciado, las partidas de ajedrez DEBEN tener el formato \textit{STG}. Además no se aceptan espacios de más, comentarios 
o anotaciones de variantes recursivas. Las etiquetas DEBEN aparecer en el orden Event, Site, Date, Round, White, Black, Result.
 Entre las etiquetas y las jugadas en formato \textit{SAN} debe haber un solo fin de línea y nada más (Nota: Se aceptan los finales de línea 
tanto de Windows como de Linux).

\subsection{ Validez de jugadas}

Existe una cierta validación de jugadas por parte del analizador, dado que se debe encontrar la pieza que se está moviendo. Esto implica que 
no pueden existir movimientos imposibles en cuanto a la naturaleza del movimiento de las piezas. Por otra parte, validaciones más complejas como 
por ejemplo restricciones para enrocar no son validadas.

\subsection{ Interfaz grafica}

Se optó por usar una interfaz gráfica que requiere librería SDL. Por lo tanto para compilar se deben instalar dos paquetes de la librería SDL; la librería 
estándar y la librería \textit{image}. En particular, en Ubuntu son las librerías \textit{libsdl1.2-dev} y \textit{libsdl-image1.2-dev}. Debido a estas 
dependencias que se agregaron, se decidió incluir el archivo compilado en la entrega final.

\newpage

\section{Descripción del desarrollo}
  Al igual que con el trabajo especial previo, al comenzar tratamos de diferenciar los distintos módulos de trabajo. 

  Una parte importante del trabajo implicaban leer el archivo .PGN, y validarlo con un \textit{parser} adecuado.
Luego, era necesario utilizar un analizador sintáctico para obtener los datos en distintos \textit{tokens}, ubicarlos en distintas
estructuras que fueron definidas con el proposito de poder procesar las jugadas según las reglas del ajedrez.
Finalmente, un módulo se encargaba del \textit{frontend} que implica la presentacion gráfica de la aplicación.

\subsection{Gramatica usada}

Notas: Los símbolos terminales son las palabars en mayúscula, que son \textit{tokens} devueltos por \textit{lex}. Los no terminales 
son aquellas en minúscula.

program $\Rightarrow$ option program | game 

option $\Rightarrow$ INITIAL_TOKEN STRING END_TOKEN  

option $\Rightarrow$ INITIAL_DATE_TOKEN INTEGER '.' INTEGER '.' INTEGER END_TOKEN 

option $\Rightarrow$ INITIAL_ROUND_TOKEN INTEGER END_TOKEN 

game $\Rightarrow$ round SPACE move SPACE move SPACE game | round SPACE move SPACE FINALRESULT | FINALRESULT | $\lambda$

round $\Rightarrow$ ROUND               

move $\Rightarrow$  castle check | normal_move check pawn_move   check 

normal_move $\Rightarrow$ PIECE COL ROW | PIECE CAPTURE COL ROW | PIECE COL COL ROW         

normal_move $\Rightarrow$ PIECE COL CAPTURE COL ROW | PIECE ROW COL ROW | PIECE ROW CAPTURE COL ROW 

pawn_move $\Rightarrow$ COL ROW | COL ROW COL ROW | COL CAPTURE COL ROW  | COL ROW CAPTURE COL ROW      
       
pawn_move $\Rightarrow$ COL ROW CROWN PIECE | COL CAPTURE COL ROW CROWN PIECE     

pawn_move $\Rightarrow$ COL ROW CAPTURE COL ROW CROWN PIECE | COL ROW COL ROW CROWN PIECE         

castle $\Rightarrow$ SHORTCASTLE   | LONGCASTLE   

check $\Rightarrow$ CHECK  | CHECKMATE  | $\lambda$

\subsection{Atributos de símbolos no terminales }

La mayoría de los \textit{tokens} devueltos por \textit{lex} tienen un valor asociado reperesntando qué es lo que se leyó. Por ejemplo, 
el atributo COL tiene el valor de la columna, ROW el valor de la fila, PIECE el carácter representando la pieza, etc.
De los símbolos no terminales, los únicos con valor son normal_move, pawn_move, move, castle, check y round. Round posee el valor númerico de la ronda, 
check un valor representando si hubo jaque, jaquemate o nada, y castle un valor representando si el enroque fue corto o largo. Normal_move, pawn_move 
y move poseen una estructura de tipo movimiento que posee muchos atributos, entre ellos, la pieza, su destino, si hubo jaque o no, enroque, etc. En las 
reescrituras del símbolo game, se llenan en esats estructuras devueltas por los movimientos asignándoles el color y se guardan en un vector para luego 
ser usadas por el \textit{front-end}.

\subsection{Lógica interna de la aplicación }

Al comenzar a armar las estructuras a usarse en la lógica interna de la aplicación, contamos como basico contar con estructuras
para representar el tablero y estructuras para representar las fichas. Después de cierto debate, decidimos apropiado que nuestro
tablero sea simplemente un vector de las 32 fichas, agregando dentro de la estructura ficha variables representando la fila y la 
columna. Esto se decidió ya que los casilleros no tienen ninguna propiedad en particular más que contener fichas o su color 
(sencillamente calculable).

Las fichas, contienen (además de su fila y su columna actual) su identificador único, su tipo de ficha, su color y si se encuentra 
viva.

Acompañadas de estas estructuras contabamos con metodos sencillos de representar los colores y los tipos de ficha (\textit{enums}).

Por otra parte, contamos con ciertas estructuras que representan lo que se lee del archivo de origen, estas son las estructuras que 
representan cada movimiento y la que representa el encabezado de cada archivo. Estas estructuras se definieron de forma sencilla a 
partir de los \textit{tokens} dados por el analizador sintactico.


\subsection{Lógica de la aplicación}

La lógica de la aplicación estaba dada por las reglas del ajedrez y simplemente consistía en corroborar que los movimientos dados
por el archivo respetasen estas reglas y ejecutar este movimiento para que tanto nuestras estructuras internas esten actualizadas 
según estos movimientos.

A continuación se presentan algunas cuestiones interesantes que debían ser validadas en este punto de la aplicación:
\begin{itemize}
 \item Detectar la ficha que se estaba moviendo.
 \item Que el movimiento sea valido para el tipo de pieza.
 \item Capturas.
 \item Coronaciones.
 \item Enroques.
\end{itemize}


\newpage

\section{Dificultades encontradas}

Al crear gramáticas con \textit{bison}  es posible encontrarse con conflictos de tipo \textit{shift-reduce}, que es justamente cuando una entrada 
en la tabla LALR(1) se puede tomar la acción de reducir o ir a otro estado (goto). En estos casos, \textit{bison} no compila. Se puede usar la 
etiqueta \textit{expect n} donde $n$ es la cantidad de conflictos presentes, para que compile. Sin embargo, lo único que hará \textit{bison} es tomar 
la acción \textit{default}  de ir a otro estado, es decir, no reducir. Esto no es siempre lo que se quiere. Idealmente uno debe evitar este tipo de 
conflictos. Se identificaron tres soluciones para resolver este tipo de conflictos:
\begin{itemize}
\item Cambiar la gramática, cambiando el lexer y agregando símbolos que toman varios de los símbolos anteriores. Esto sería darle más trabajo al lexer.
También  se puede hacer uso de los estados que usa el lexer, ya sea los estados izquierda (aquellos encerrados por $<$ o $>$) o estados de tipo derecha (
pedir, por ejemplo, la cadena $abd|eff$, que signfica pedir la cadena $abd$ si y solo si está seguida por la cadena $eff$). 
\item Cambiar el parser (\textit{yacc}) y hacer reglas eliminando recursiones  o reglas dispuestas distintas para que no conflictúen. Esto sería darle 
más trabajo al parser.
\item Manejar el problema en la parte semántica, es decir, resolverlo en el código $C$. Esto es darle más trabajo al lenguaje que se está usando.
\end{itemize}

En el desarrollo del trabajo se intentó dividir el trabajo entre estas tres herramientas en una manera que el trabajo de los tres no sea ni trivial ni 
extremadamente complicado, priorizando la prolijidad del código. Finalmente se logró que la gramática resultanto no tenga conflictos.


\newpage

\section{Futuras Extensiones}

\subsection{ Consideraciones }

Como ya se ha explicado en una sección anterior, se asumen muchas cosas en cuanto al formato de entrada del archivo. Se podría modificar la gramática para 
ser más laxos en cuanto a qué se acepta y qué no. Por ejemplo, se podrían aceptar las opciones en cualquier orden, permitir espacios de más, etc. Desde ya 
esto agregaría una complejidad apreciable a la gramática usada y a las reglas de sus producciones.

\subsection{ Formato }

Se podrían aceptar comentarios o más opciones que las siete requeridas. Además, sería interesante poder mostrar comentarios en sus jugadas respectivas. 

\subsection{ Retroceso }

El programa está diseñado para ir solo adelante con las jugadas. Se podría implementar que con una cierta tecla se vaya hacia adelante y con otra hacia atrás, 
de esa manera se podría hacer un seguimiento del partido con mayor libertad.

\subsection{ Marcado de movimientos }

Se podría marcar el casillero destino y fuente de la última pieza  movida de tal manera de simplificar el seguimiento del partido.
   
\end{document}

