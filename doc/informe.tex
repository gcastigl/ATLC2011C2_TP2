\documentclass[a4paper,10pt]{article}

\usepackage[utf8]{inputenc}
\usepackage{t1enc}
\usepackage[spanish]{babel}
\usepackage[pdftex,usenames,dvipsnames]{color}
\usepackage[pdftex]{graphicx}
\usepackage{amsmath}
\usepackage{amsfonts}
\usepackage{amssymb}
\usepackage[table]{xcolor}
\usepackage[small,bf]{caption}
\usepackage{float}
\usepackage{subfig}
\usepackage{listings}
\usepackage{bm}
\usepackage{times}

\setcounter{secnumdepth}{5}

\begin{document}

\begin{titlepage}
	\thispagestyle{empty}
	\begin{center}
		\includegraphics[scale=0.7]{./images/itba.jpg}
		\vfill
		\Huge{Autómatas, Teoria de Lenguajes y Compiladores}\\
		\vspace{1cm}
		\huge{Trabajo Práctico Especial 2} \\
		\vspace{0.3cm}
		\huge{Title}
	\end{center}
	\vspace{2cm}
	\large{
		\begin{tabular}{lcr}
			Castiglione, Gonzalo & & 49138 \\
			Susnisky, Darío & & 50592 \\
			Ordano, Esteban & & 50753 \\
			Sturla, Martín & & 50684 \\
			\\ 
		\end{tabular}
	}
	\vfill
	\flushright{\today}
\end{titlepage}

\newpage

%%%%%%%%%%%%%%%%%%%%%%%%%%%%%%%%%%
%%%%%%%%% begin CONTENT %%%%%%%%%%
%%%%%%%%%%%%%%%%%%%%%%%%%%%%%%%%%%

	\thispagestyle{empty}
\tableofcontents

\newpage

\setcounter{page}{1}

\newpage

\section{Resumen}
El trabajo práctico consistió en generar un analizador sintáctico de partidas de ajedrez en formato \textit{PGN} 
para luego mostrar un tablero y poder hacer un seguimiento del partido.
\newpage

\section{Consideraciones realizadas}

\subsection{Formato PGN}

Como indica el enunciado, las partidas de ajedrez DEBEN tener el formato \textit{STG}. Además no se aceptan espacios de más, comentarios 
o anotaciones de variantes recursivas. Las etiquetas DEBEN aparecer en el orden Event, Site, Date, Round, White, Black, Result.
 Entre las etiquetas y las jugadas en formato \textit{SAN} debe haber un solo fin de línea y nada más (Nota: Se aceptan los finales de línea 
tanto de Windows como de Linux).

\subsection{ Validez de jugadas}

Existe una cierta validación de jugadas por parte del analizador, dado que se debe encontrar la pieza que se está moviendo. Esto implica que 
no pueden existir movimientos imposibles en cuanto a la naturaleza del movimiento de las piezas. Por otra parte, validaciones más complejas como 
por ejemplo restricciones para enrocar no son validadas.

\subsection{ Interfaz grafica}

Se optó por usar una interfaz gráfica que requiere librería SDL. Por lo tanto para compilar se deben instalar dos paquetes de la librería SDL; la librería 
estándar y la librería \textit{image}. En particular, en Ubuntu son las librerías \textit{libsdl1.2-dev} y \texgit{libsdl-image1.2-dev}. Debido a estas 
dependencias que se agregaron, se decidió incluir el archivo compilado en la entrega final.

\newpage

\section{Descripción del desarrollo}
  
\subsection{Gramatica usada}

\newpage

\section{Dificultades encontradas}
\newpage

\section{Futuras Extensiones}
   
\end{document}

