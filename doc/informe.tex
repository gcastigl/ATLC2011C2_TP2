\documentclass[a4paper,10pt]{article}

\usepackage[utf8]{inputenc}
\usepackage{t1enc}
\usepackage[spanish]{babel}
\usepackage[pdftex,usenames,dvipsnames]{color}
\usepackage[pdftex]{graphicx}
\usepackage{amsmath}
\usepackage{amsfonts}
\usepackage{amssymb}
\usepackage[table]{xcolor}
\usepackage[small,bf]{caption}
\usepackage{float}
\usepackage{subfig}
\usepackage{listings}
\usepackage{bm}
\usepackage{times}

\setcounter{secnumdepth}{5}

\begin{document}

\begin{titlepage}
	\thispagestyle{empty}
	\begin{center}
		\includegraphics[scale=0.7]{./images/itba.jpg}
		\vfill
		\Huge{Autómatas, Teoria de Lenguajes y Compiladores}\\
		\vspace{1cm}
		\huge{Trabajo Práctico Especial 2} \\
		\vspace{0.3cm}
		\huge{Title}
	\end{center}
	\vspace{2cm}
	\large{
		\begin{tabular}{lcr}
			Castiglione, Gonzalo & & 49138 \\
			Susnisky, Darío & & 50592 \\
			Ordano, Esteban & & 50753 \\
			Sturla, Martín & & 50684 \\
			\\ 
		\end{tabular}
	}
	\vfill
	\flushright{\today}
\end{titlepage}

\newpage

%%%%%%%%%%%%%%%%%%%%%%%%%%%%%%%%%%
%%%%%%%%% begin CONTENT %%%%%%%%%%
%%%%%%%%%%%%%%%%%%%%%%%%%%%%%%%%%%

	\thispagestyle{empty}
\tableofcontents

\newpage

\setcounter{page}{1}

\newpage

\section{Resumen}
El trabajo práctico consistia en poder leer un archivo .PGN (archivos estándar para partidas de ajedrez), validarlos y mostrar
la jugada. Para esto, era necesario leer e interpretar los archivos de entrada con archivos de \textit{lex} y \textit{bison}
 utilizando analizadores sintácticos muy similares a los vistos en clase.

\newpage

\section{Consideraciones realizadas}
    No hubo consideraciones externas hechas ya que en caso de errores en los archivos de entradas era nuestro trabajo
     detectarlos. Por otra parte, dentro de la lógica del programa si se valida que la partida de ajedrez sea coherente 
      (se valida lo más posible que los movimientos sean coherentes con las reglas del juego y con el estado actual del tablero). \\

\newpage

\section{Descripción del desarrollo}
    Al igual que con el trabajo especial previo, al comenzar tratamos de diferenciar los distintos módulos de trabajo. 

 Una parte importante del trabajo implicaban leer el archivo .PGN, y validarlo con un \textit{parser} adecuado.
Luego, era necesario utilizar un analizador sintáctico para obtener los datos en distintos \textit{tokens}, ubicarlos en distintas
estructuras que fueron definidas con el proposito de poder procesar las jugadas según las reglas del ajedrez.
Finalmente, un módulo se encargaba del \textit{frontend} que implica la presentacion gráfica de la aplicación.

\subsection{Parser}

\subsection{Gramatica usada}

\subsection{Estructuras elegidas}

\subsection{Lógica interna de la aplicación}

\subsection{Presentación gráfica}

\newpage

\section{Dificultades encontradas}
\newpage

\section{Futuras Extensiones}
   
\end{document}

